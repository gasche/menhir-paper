\documentclass[twoside,a4paper]{article}
\usepackage{actes}
\usepackage[T1]{fontenc}
\usepackage[english]{babel}
%% Title page...
\title{Rolling Menhir: error messages and recovery for type-safe LR parsers}

\author{Frédéric Bour}%$^1$}
%% Headers for other pages...
\titlehead{Rolling Menhir} % for odd pages
\authorhead{F. Bour}     % for even pages
% ?!
%\affiliation{\begin{tabular}{rr}        % for institute(s) 
%\\ 1:  Laboratoire d'Informatique de l'\'Ecole UniTechnique,
%\\     99999 Petit Oiseau CEDEX, France
%\\     {\tt mchulott@info.unitechnique.fr} 
%\end{tabular}}
                                                                
\begin{document}
\setcounter{page}{1}
\maketitle
% Here is the place for your text...

\section{Introduction}

% Pour spécifier une grammaire de manière déclarative et pouvoir analyser le
% langage, LR est une solution satisfaisante.
% Menhir en est une implémentation pour OCaml.
% Problème : répond au problème de traiter une entrée correcte. Il est parfois
%  inévitable de travailler avec des entrées partielles ou erronnées. C'est le
%  cas dans Merlin, un outil pour analyser du code OCaml depuis un éditeur.
% LR a une mauvaise réputation dans ces situations. 
% - Yacc, un token error et une logique de récupération fixe et peu flexible.
% - Merr, permet de générer des messages d'erreurs en identifiant certaines
%   situations.
% Notre solution: externaliser la gestion des erreurs (plus de logique dans
% Menhir), en proposant des primitives:
% - pour introspecter l'état du parser, et notamment interpréter cet état en
%   terme de grammaire et non de l'automate sous-jacent.
% - pour manipuler l'état du parser, outside of classical parsing, en
%   conservant une bonne formation
% L'implémentation est actuellement utilisée en production pour Merlin, offrant
% des résultats acceptables pour le langage OCaml.

\section{Well-typed interface for incremental parsing}

% Prérequis pour la suite du travail : du parsing pur.

% Menhir a deux backends : natif & table.
% Natif utilise la stack système, table utilise une pile réifiée.
% Cependant la seule interface exposée est héritée de Yacc : l'utilisateur
% fournit un lexer au parser qui prend le contrôle, jusqu'à un éventuel succès
% ou une erreur.

% Le travail étend uniquement le backend table. (Natif: conçu pour la vitesse,
% des optimisations rendent la représentation non-uniforme, très peu flexible)

\subsection{Pure parsing}

% Interface pure:
%   type parser
%   type status = Reduce | Shift | Shift_and_feed | Accept | Reject

%   val reduce : parser -> parser * status
%   val shift : parser -> parser * status
%   val shift_and_feed : position * token * position 
%                      -> parser -> parser * status 

% Philosophie: une interface très bas-niveau, l'objectif est uniquement
% d'extraire la boucle d'interprétation pour rendre le contrôle à
% l'utilisateur, en préservant autant que possible les performances
% (en pratique : complexité inchangée, allocations supplémentaires à la place
% de mutations, TODO benchmark).

% Les parsers étant pures, up-to user provided semantic actions, l'analyse
% peut-être résumée depuis n'importe quel parser existant.

% 
\paragraph{Initialization}

\paragraph{Getting the results}

\subsection{Typed representation of the stack}

\section{Exploring the stack}

\subsection{LR Itemset}

\paragraph{Annotations}

\subsection{Producing error messages}

\subsection{Recovering from errors}

\section{Experimental results}

\section{Conclusion}


\paragraph{Future work}

% L'interface exposée est composée de primitives de bas niveau, rechercher un
% jeu de combinateurs adapté à une spécification de plus haut-niveau.

% L'heuristique basée sur l'indentation a été utile pour valider rapidement les
% choix de conception, en restant générique et en offrant une qualité de
% reconstruction acceptable.   Cependant, nous espérons pouvoir améliorer
% sensiblement la reconstruction, en tirant plus finement partie de la
% structure grammaticale du langage.

% Notre objectif a plus moyen terme est un toolkit pour travailler avec des
% entrées incorrectes d'un langage LR, en ne demandant qu'un faible
% investissement au développeur.

% Do NOT FORGET to include your bibliography for submission
\bibliographystyle{abbrv}
\bibliography{myreferences}
                                                                
\end{document}
