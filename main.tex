\documentclass[twoside,a4paper]{article}
\usepackage{actes}
\usepackage[T1]{fontenc}
\usepackage[english]{babel}
%% Title page...
\title{Rolling Menhir: error messages and recovery for type-safe LR parsers}

\author{Frédéric Bour}%$^1$}
%% Headers for other pages...
\titlehead{Rolling Menhir} % for odd pages
\authorhead{F. Bour}     % for even pages
% ?!
%\affiliation{\begin{tabular}{rr}        % for institute(s) 
%\\ 1:  Laboratoire d'Informatique de l'\'Ecole UniTechnique,
%\\     99999 Petit Oiseau CEDEX, France
%\\     {\tt mchulott@info.unitechnique.fr} 
%\end{tabular}}
                                                                
\begin{document}
\setcounter{page}{1}
\maketitle
% Here is the place for your text...

\section{Introduction}

% Pour spécifier une grammaire de manière déclarative et pouvoir analyser le
% langage, LR est une solution satisfaisante.
% Menhir en est une implémentation pour OCaml.

LR parser generators offer a decent solution to specify a grammar in a
declarative fashion.  Menhir is a modern implementation of such a generator for
the OCaml programming langage.

% Problème : répond au problème de traiter une entrée correcte. Il est parfois
%  inévitable de travailler avec des entrées partielles ou erronnées. C'est le
%  cas dans Merlin, un outil pour analyser du code OCaml depuis un éditeur.

However, the focus is on efficient parsing of correct inputs.  One is often
faced with the need to process incomplete or incorrect inputs, as is the case
in IDE and other developer tools.

% LR a une mauvaise réputation dans ces situations. 

While it would be useful to be able to reuse grammars in those situations, LR
earned a bad reputation: limited support from the existing tools,
implementation details exposed to the developer (automaton states).

% - Yacc, un token error et une logique de récupération fixe et peu flexible.
% - Merr, permet de générer des messages d'erreurs en identifiant certaines
%   situations.

\paragraph{Yacc error recovery mode.}
The original Yacc provides a form of error processing and recovery. A fake {\em
error} token is introduced. When an invalid token is encountered, rather than
stopping the parse, the token is replaced by the {\em error} one and parse is
resumed.

Therefore by adding new rules dealing with the {\em error} symbol, it is
possible to catch and recover from an unexpected situation.  Furthermore, if no
rules accept {\em error} at the current state, Yacc will pop states from the
stack until finding one dealing with {\em error} or failing definitively.

However, inserting the error rules require great care from the developer and to
keep a mental model of the parser execution.  In particular, it is hard to
prevent conflicts between error rules.

\paragraph{Merr error messages.}
More recently, Merr offered a more user-friendly solution focusing on
generating error messages, indexed by the automaton state at error point. 

The developer is expected to provide a list of erroneous inputs together with
the associated error messages.  Merr will run the parser on each input, and
produce a mapping from the number of the state reached just before failing to
the error message.

\paragraph{Our solution} generalizes on both solutions by extracting the
recovery logic out of the generated parser.  A new programming interface is
exposed to the developer, with primitives allowing to:

\begin{itemize}
  \item introspect the parser state, the corresponding LR itemset to work at
    the grammar level, and the semantic values en the stack;
  \item manipulate the state of the parser, to introduce strategies outside of
    normal parsing, while keeping the state well-formed.
\end{itemize}

Both Yacc \& Merr strategies can be reimplemented as special cases with those
primitives, without specific support from Menhir.

We also implemented other strategies, to demonstrate applicability of our
approach in practical cases.  Those are used in production with satisfying
results in Merlin, a developer tool providing interactive type-checking and
completion for OCaml.

% Notre solution: externaliser la gestion des erreurs (plus de logique dans
% Menhir), en proposant des primitives:
% - pour introspecter l'état du parser, et notamment interpréter cet état en
%   terme de grammaire et non de l'automate sous-jacent.
% - pour manipuler l'état du parser, outside of classical parsing, en
%   conservant une bonne formation
% L'implémentation est actuellement utilisée en production pour Merlin, offrant
% des résultats acceptables pour le langage OCaml.

\section{Well-typed interface for incremental parsing}

% Prérequis pour la suite du travail : du parsing pur.

% Menhir a deux backends : natif & table.
% Natif utilise la stack système, table utilise une pile réifiée.

Menhir offers two backends.  The first one, native, directly uses the system
stack to encode the LR one in an optimized fashion.  This provides great
performance but makes it highly unpractical to extract information from the
stack.  The second one, table, reifies the stack in an ML-value, making it easy
to process from the outside.

% Cependant la seule interface exposée est héritée de Yacc : l'utilisateur
% fournit un lexer au parser qui prend le contrôle, jusqu'à un éventuel succès
% ou une erreur.

However, the interface exposed in the generated files is directly inherited
from Yacc: the user provides a lexer by the mean of an impure function, then
the parser grabs control until eventually producing a result or failing.

  val entry\_point: (Lexing.lexbuf -> token) -> Lexing.lexbuf -> t

As a prerequisite, we will extend the table backend to let the user takes
control of parsing flow, switch the internal representation to a purely
functional one and expose instances of the parser as ML-values.

Inversion of control and immutable representation of the parser will greatly
help introspection and manipulation.

% Le travail étend uniquement le backend table. (Natif: conçu pour la vitesse,
% des optimisations rendent la représentation non-uniforme, très peu flexible)

\subsection{Pure parsing}

% Interface pure:
%   type parser
%   type status = Reduce | Shift | Shift_and_feed | Accept | Reject

%   val reduce : parser -> parser * status
%   val shift : parser -> parser * status
%   val shift_and_feed : position * token * position 
%                      -> parser -> parser * status 

% Philosophie: une interface très bas-niveau, l'objectif est uniquement
% d'extraire la boucle d'interprétation pour rendre le contrôle à
% l'utilisateur, en préservant autant que possible les performances
% (en pratique : complexité inchangée, allocations supplémentaires à la place
% de mutations, TODO benchmark).

% Les parsers étant purs, up-to user provided semantic actions, l'analyse
% peut-être résumée depuis n'importe quel parser existant.

\paragraph{Initialization}

%   type 'a initial
%   val initiliaze : 'a initial -> position * token * position -> parser

\paragraph{Getting the results}

%   val get_results : 'a initial -> parser -> 'a

% Use-case for interactive parsing:
% - in merlin, when the user modifies earlier part of the file,
%   parsing is resumed from the token immediately before the modification
% - in REPL/shells, it is often useful to detect whether current phrase is
%   well-formed, to either execute immediately or continue prompt on next line
% It can be done easily with the pure representation by sending a specific 
% token, executing if the parser accepts or backtracking if it rejects.

\subsection{Typed representation of the stack}

% Pottier : GADT pour toute la pile, typer les transitions.
% Trop fin, difficile à générer, difficile à manipuler

% Alternative : les valeurs sémantiques sont issues soit des terminaux,
% soit des non-terminaux.
% Utiliser des GADTs dont les constructeurs indiquent le type des valeurs
% sémantiques des terminaux et des non-terminaux.
% Embarquer le tag avec la valeur sémantique dans la pile.

% On conserve type-safety, on n'a pas le tes d'exhaustivité (runtime assertion)
% le coût avec la représentation actuelle est de 3 mots plus une indirection.
% Il devient possible d'introspecter la pile, using terminals and non-terminals
% from the grammar definitiion.

\section{Exploring the stack}

\subsection{LR Itemset}

\paragraph{Annotations}

\subsection{Producing error messages}

\subsection{Recovering from errors}

\section{Experimental results}

\section{Conclusion}


\paragraph{Future work}

% L'interface exposée est composée de primitives de bas niveau, rechercher un
% jeu de combinateurs adapté à une spécification de plus haut-niveau.

% L'heuristique basée sur l'indentation a été utile pour valider rapidement les
% choix de conception, en restant générique et en offrant une qualité de
% reconstruction acceptable.   Cependant, nous espérons pouvoir améliorer
% sensiblement la reconstruction, en tirant plus finement partie de la
% structure grammaticale du langage.

% Notre objectif a plus moyen terme est un toolkit pour travailler avec des
% entrées incorrectes d'un langage LR, en ne demandant qu'un faible
% investissement au développeur.

% Do NOT FORGET to include your bibliography for submission
\bibliographystyle{abbrv}
\bibliography{myreferences}

\end{document}
